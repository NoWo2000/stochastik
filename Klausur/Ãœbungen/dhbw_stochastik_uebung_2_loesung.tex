\documentclass[a4paper,12pt]{article}

%\usepackage[latin1]{inputenc}
\usepackage{amsfonts}
\usepackage{amsmath}
\usepackage{amssymb}
\usepackage{amsthm}
\usepackage{color}
\usepackage[ngerman]{babel}
\usepackage[pdftex]{graphicx}
%\usepackage[T1]{fontenc}
\usepackage{gauss}

\pagestyle{empty}
\usepackage[utf8]{inputenc}
\usepackage{listings}
\definecolor{Brown}{cmyk}{0,0.81,1,0.60}
\definecolor{OliveGreen}{cmyk}{0.64,0,0.95,0.40}
\definecolor{CadetBlue}{cmyk}{0.62,0.57,0.23,0}
\definecolor{lightlightgray}{gray}{0.9}
\lstset{
    language=C,                             % Code langugage
    basicstyle=\rm\ttfamily,                % Code font, Examples: \footnotesize, \ttfamily
    keywordstyle=\color{OliveGreen},        % Keywords font ('*' = uppercase)
    commentstyle=\color{gray},              % Comments font
    numbers=left,                           % Line nums position
    numberstyle=\tiny,                      % Line-numbers fonts
    stepnumber=1,                           % Step between two line-numbers
    numbersep=5pt,                          % How far are line-numbers from code
    backgroundcolor=\color{lightlightgray}, % Choose background color
    frame=none,                             % A frame around the code
    tabsize=2,                              % Default tab size
    captionpos=b,                           % Caption-position = bottom
    breaklines=true,                        % Automatic line breaking?
    breakatwhitespace=false,                % Automatic breaks only at whitespace?
    showspaces=false,                       % Dont make spaces visible
    showtabs=false,                         % Dont make tabls visible
    %columns=flexible,                       % Column format
    %morekeywords={someword, otherword},     % specific keywords
}



%\topmargin20mm
\oddsidemargin0mm
\parindent0mm
\parskip2mm
\textheight24cm
\textwidth15.8cm
\unitlength1mm
\usepackage{pdfpages}

\begin{document}



{\bf Aufgabe 1: Kombinatorik}

In einem Raum gibt es acht Lampen, die unabhängig voneinander ein- und ausgeschaltet werden können.
Wie viele Beleuchtungsarten gibt es

{\bf a) (12,5 Punkte) }

wenn fünf Lampen brennen sollen?

{\bf b) (7,5 Punkte)}

wenn mindestens fünf Lampen brennen sollen?

{\bf Lösung}

Für die Auswahl von $n$  Lampen aus $N= 8$  ohne Beachtung der Reihenfolge gibt es 
\begin{align*}
M(n):= \begin{pmatrix} 8 \\ n\end{pmatrix}
\end{align*}
Möglichkeiten (6P).


{\bf a) }
$M(5) = \begin{pmatrix} 8 \\ 5\end{pmatrix}  \; (2.5P)= \frac{8!}{5! (8-5)!}  \;(1P) =  \frac{8!}{5! (3)!} \; (1P) =  \frac{8 \cdot 7 \cdot 6}{3 \cdot 2} \; (1P) = 8 \cdot 7 = 56 \;(1P)$

{\bf b) }
Für mindestens fünf brennende Lampen gibt es $M(5) + M(6) + M(7)  + M(8)  \; (3.5P)= 56 + 28 + 8 + 1  \; (3P)= 93  \; (1P)$ Möglichkeiten.
\hspace{10mm}



{\bf Aufgabe 2: Bedingte Wahrscheinlichkeiten}

In einer Fabrik werden die produzierten Werkstücke vor der Auslieferung überprüft. Hierfür werden für jedes Werkstück  hintereinander zwei Funktionstest durchgeführt.  Die Wahrscheinlichkeit, dass ein Werkstück beide Tests besteht, betrage $0,55$. Die Wahrscheinlichkeit, dass ein Werkstück  den ersten Test besteht betrage $0,72$. 

{\bf a) (15 Punkte) }

Berechnen Sie die Wahrscheinlichkeit, dass ein Werkstück den zweiten Test besteht, wenn er bereits den ersten bestanden hat.

{\bf b) (5 Punkt) }

Angenommen, die beiden Tests sind stochastisch unabhängig. Wie hoch ist dann die Wahrscheinlichkeit, dass ein Werkstück den zweiten Test besteht?

{\bf Lösung}

Wahrscheinlichkeit für  Test $T_1$  ist $P(T_1) = 0,72  \; (4P)$. Wahrscheinlichkeit für  Test $T_1$ und Test $T_2$ ist $P(T_1 \cap T_2) = 0,55  \; (4P)$.

{\bf a) }
$P(T_2 | T_1) = \frac{P(T_1 \cap T_2)}{P(T_1)}  \; (6P)= \frac{0,55} {0,72} \approx 0,7639  \; (1P)$.


{\bf b) }
Sind $T_1$ und $T_2$ unabhängig, gilt  $P(T_2) = P(T_2 | T_1)  \; (4P) = 0,7639  \; (1P)$.

\hspace{10mm}

{\bf Aufgabe 3: Wahrscheinlichkeitsraum und Verteilungen}

{\bf a) (10 Punkte) }

Geben Sie die Axiome für einen Wahrscheinlichkeitsraum an. 

{\bf b) (10 Punkte) }

Geben Sie eine Konstante $c  \in \mathbb{R}$ an, so dass die Funktion 
\begin{align*}
f(x) = \begin{cases} c x^2 \text{ für }  0\leq x \leq 1 \\ 0 \text{ sonst}\end{cases}
\end{align*}
ein Dichte auf $\mathbb{R}$ definiert.


{\bf Lösung}

{\bf a) }
Ein Wahrscheinlichkeitsraum ist ein Tripel $(\Omega, \mathcal{A}, P) $ bestehend aus der Grundmenge $\Omega  \; (1P)$, einer $\sigma$-Algebra $\mathcal{A} \subset  \mathcal{P}(\Omega)  \; (1P)$ und einer Abbildung
$P : \mathcal{A} \to [0,1]  \; (1P)$
\begin{align*}
(i) & \; P(\Omega) = 1   \; (3P)\\
(ii) & \;  P \biggl(  \bigcup_i A_i  \biggr) = \sum_i P(A_i)  \; (3P), \text{ mit } A_i \cap A_j = \emptyset \text{ für } i \neq j  \; (1P)
\end{align*}

{\bf b) }
Integration über eine Dichte muss $1$ ergeben (4P).  
$\int_0^1 x^2 \; dx = \frac{1}{3}  \; (4P) \Rightarrow c = 3 \; (2P)$.


{\bf Aufgabe 4: Zufallsvariable und Erwartungswert}

{\bf a) (10  Punkte) }

Die Zufallsvariablen $X_1$ und $X_2$ seien stochastisch unabhängig   und im Intervall $[0,1]$ gleichverteilt.
Berechnen Sie den Erwartungswert der Zufallsvariablen $Y = X_1 \cdot (X_2 - X_1)$.

{\bf b) (10 Punkte) }

Formulieren Sie das schwache Gesetz  der grossen Zahlen und erläutern Sie die Aussage.



{\bf Lösung}

{\bf a) }
Gleichverteilung auf  $[0,1] \Rightarrow  \mathbb{E}(X_1) =  \mathbb{E}(X_2) = \frac{1}{2} \; (2P)$. und $ \mathbb{E}(X_1^2) = \frac{1}{3} \; (2P)$. Da $X_1$ und $X_2$ unabhängig $\Rightarrow \mathbb{E}(X_1 X_2) =  \mathbb{E}(X_1)  \mathbb{E}(X_2)  = \frac{1}{4} \; (2P)$.  Damit ist $\mathbb{E}(Y) = \mathbb{E}( X_1 \cdot (X_2 - X_1)) \; = \mathbb{E}( X_1 X_2 - X_1^2)  \; (1P) = \mathbb{E}( X_1 X_2) - \mathbb{E}(X_1^2)  \; (2P) =  \frac{1}{4} - \frac{1}{3} = \frac{-1}{12} \; (1P)$. 


{\bf b) }
Seien $\{X_i  \}_i: \Omega \to \mathbb{R}$ unabhängige (1P), reelle Zufallsvariablen mit $\mathbb{E}(X_i) = \mu < \infty \; (1P)$ und $\mathbb{V}(X_i) = \sigma < \infty \; (1P)$. Dann gilt:
\begin{align*}
P \bigl  ( \bigl | \frac{1}{n} \sum_{i=1}^{n} X_i - \mu \bigr |  \geq \epsilon \bigr) \leq \frac{\sigma}{ n \cdot \epsilon^2} \; \; \underset{n \to \infty}{\longrightarrow} 0 \; (4P)
\end{align*}
Das schwache Gesetz der Grossen zahlen besagt, dass das arithmetische Mittel  einer grossen Stichproben (1P) einer Zufallsvariable mit einer beliebig kleinen, vorgegebenen Wahrscheinlichkeit (1P) dem Erwartungswert der Zufallsvariable entspricht (1P). 

\hspace{10mm}




{\bf Aufgabe 5: Hypothesentest  (20 Punkte)}

In der Zeitung lesen Sie, dass es im sonnigen Mannheim im Schnitt höchstens an 10 von 100 Tagen regnet. Sie möchten diese Hypothese überprüfen und beobachten hierfür an $n=20$ Tagen das Wetter. Erstellen Sie einen geeigneten Hypothesentest. 
Wie sollte Ihre Entscheidungsregel dann lauten, wenn Sie sicherstellen möchten, dass die Hypothese mit einer Irrtumswahrscheinlichkeit von $0.05$ unberechtigter Weise abgelehnt wird.

{\bf Lösung}

Wir wählen das Binomialmodell $(\mathcal{X}= \{ 0, \cdots, n\},  P(\mathcal{X}), P_\rho = B_{n, \rho} : \rho \in \Theta =  [0 ,1] ) \; (2P)$, wobei $P_\rho (\{k \})=  B_{n, \rho}(\{k\}) = \begin{pmatrix} n \\ k \end{pmatrix} \rho^k (1-\rho)^{n-k} $.  Wir testen die Nullhypothese $H_0: \rho \leq 0.1 \; (2P)$ gegen die Alternative $H_1: \rho > 0.1 \; (2P)$ Als Signifikanzniveau wählen wir $\alpha = 0.05 \; (2P)$ und als Statistik wählen wir die Identität $T(x) = x$. Als Testfunktion wählen wir $\varphi = 1_{ \{ c, \cdots, n \}} \; (2P)$ und müssen $c \in \mathcal{X}$ so wählen, dass $\sup_{\rho \in \Theta_0} \mathbb{E}(\varphi )  < \alpha \; (2P)$ gilt.  Wir ermitteln also $c-1$ als das $0.05$-Fraktil der Binomialverteilung  (2P) mit $20$ Freiheitsgraden (2P) und damit $c = 5$ (2P), da $P_{0,1} (x \geq 4) > 0,05 $ und  $P_{0,1} (x \geq 5) = 1 -P_{0,1} (4) \approx 1 - 0,9568 = 0,0432 < 0,05$ (2P).

{\end{document}
