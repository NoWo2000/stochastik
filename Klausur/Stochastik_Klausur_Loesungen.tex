\documentclass[a4paper,13pt]{scrartcl}

\usepackage[utf8]{inputenc}
\usepackage{amsfonts}
\usepackage{amsmath}
\usepackage{amssymb}
\usepackage{amsthm}
\usepackage{enumitem}
\usepackage{hyperref}
\usepackage[all]{hypcap}
\hypersetup{
        colorlinks = true, % comment this to make xdvi work
        linkcolor  = blue,
        citecolor  = red,
        filecolor  = Gold,
        urlcolor   = blue,
        pdfborder  = {0 0 0} 
}
\usepackage{color}
\usepackage[ngerman]{babel}
\usepackage[pdftex]{graphicx}
%\usepackage[T1]{fontenc}
\usepackage{graphicx}
\pagestyle{empty}

%\topmargin20mm
\oddsidemargin0mm
\parindent0mm
\parskip2mm
\textheight25.5cm
\textwidth15.8cm
\unitlength1mm

\newcommand{\exercise}{\vspace*{0.2cm}
\stepcounter{aufgabe}
\noindent
\textbf{Aufgabe \arabic{aufgabe}}: }
\newcounter{aufgabe}

\newcommand{\solution}{\vspace*{0.2cm}
\noindent
\textbf{Lösung}: }



\begin{document}
\section*{\large  Aufgaben Übungsblätter}
\hrule
\hrule
\vspace{4mm}

\exercise
Beim Lottospiel werden ohne Zurücklegen $6$ Zahlen aus $49$ gezogen. Berechnen Sie die folgenden Wahrscheinlichkeiten:
\begin{enumerate}[label=(\alph*)]
\item Alle $6$ Gewinnzahlen richtig zu tippen.
\item Genau $5$ richtige Gewinnzahlen zu tippen.
\item Mindestens $3$ richtige Gewinnzahlen zu tippen.
\item Alle $6$ Gewinnzahlen sind gerade.
\end{enumerate}
\vspace{4mm}

\solution
 Der Grundraum beim Lottospiel ist eine Kombination ohne Beachtung der Reihenfolge
$\Omega = \{ (z_1, \dots,  z_6) \; | \;  z_i \in \{1, \cdots, 49 \} \land z_1 < z_2 < \cdots < z_6 \}$ und damit ist $$| \Omega| = \begin{pmatrix} 
    49 \\ 6 \end{pmatrix} = 13983816$$
Die Anzahl $A_i$ an Kombinationen $i$  Zahlen aus $6$ richtigen zu tippen ist $$ \begin{pmatrix} 
    6 \\ i \end{pmatrix} \cdot \begin{pmatrix} 
    43   \\ 6 -i \end{pmatrix}$$ ($i$ Zahlen von $6$ richtig Tippen mal $6-i$ Zahlen von den falschen $49-6$ Zahlen zu tippen)\\
Die Lösungen sind nun:
\begin{enumerate}[label=(\alph*)]
\item $ | A_6 |  = 1 \Rightarrow P(A_6) = \frac{1}{|\Omega|}  \approx 7,14 \cdot 10^{-8}$
\item $ | A_5 |  =  \begin{pmatrix} 
    6 \\ 5 \end{pmatrix} \cdot \begin{pmatrix} 
    43   \\ 1 \end{pmatrix} = 258 \Rightarrow P(A_5) = \frac{258}{|\Omega|}   \approx 1,845 \cdot 10^{-5}$
\item $B = \{ \text{mindestens 3 richtig}\} = A_6 \cup A_5 \cup A_4 \cup A_3$.\\
$$P(B) = P (A_6 \cup A_5 \cup A_4 \cup A_3)= P(A_6) + P(A_5) + P(A_4) + P(A_3)$$
(Da $A_i$ paarweise disjunkt)\\
 $$ | A_4 |  =  \begin{pmatrix} 
    6 \\ 4 \end{pmatrix} \cdot \begin{pmatrix} 
    43   \\ 2 \end{pmatrix} = 13545 \Rightarrow P(A_4) = \frac{13545}{|\Omega|}   \approx 9,69 \cdot 10^{-4}$$
    
 $$ | A_3 |  =  \begin{pmatrix} 
    6 \\ 3 \end{pmatrix} \cdot \begin{pmatrix} 
    43   \\ 3 \end{pmatrix} = 246820 \Rightarrow P(A_4) = \frac{246820}{|\Omega|}   \approx 0,01765$$
\\$\Rightarrow P(B) \approx 0,018637545$
\item Es gibt $24$ gerade Zahlen zwischen $1$ und $49$ und damit $ \begin{pmatrix} 
    24 \\ 6 \end{pmatrix}$ Möglichkeiten, dass alle Gewinnzahlen gerade sind. Damit ist $$P(\texttt{nur gerade Gewinnzahlen}) = \frac{\begin{pmatrix} 
    24 \\ 6 \end{pmatrix}}{\begin{pmatrix} 
    49 \\ 6 \end{pmatrix}} \approx 0,00963$$
\end{enumerate}
\vspace{8mm}


\exercise
In einem Raum gibt es acht Lampen, die unabhängig voneinander ein- und ausgeschaltet werden können.
Wie viele Beleuchtungsarten gibt es
\begin{enumerate}[label=(\alph*)]
\item wenn fünf Lampen brennen sollen?
\item wenn mindestens fünf Lampen brennen sollen?
\end{enumerate}
\vspace{4mm}

\solution
Für die Auswahl von $n$  Lampen aus $N= 8$  ohne Beachtung der Reihenfolge gibt es 
\begin{align*}
M(n):= \begin{pmatrix} 8 \\ n\end{pmatrix}
\end{align*}
Möglichkeiten.
\begin{enumerate}[label=(\alph*)]
\item $M(5) = \begin{pmatrix} 8 \\ 5\end{pmatrix}  \; = \frac{8!}{5! (8-5)!}  \; =  \frac{8!}{5! (3)!} \;  =  \frac{8 \cdot 7 \cdot 6}{3 \cdot 2} \;  = 8 \cdot 7 = 56$
\item Für mindestens fünf brennende Lampen gibt es $M(5) + M(6) + M(7)  + M(8)  \; = 56 + 28 + 8 + 1  \; = 93$ Möglichkeiten.
\end{enumerate}
\vspace{8mm}


\exercise
Sie geben eine Party und laden 10 Leute ein.
\begin{enumerate}[label=(\alph*)]
\item Wie viele Möglichkeiten gibt es, die 10 Gäste an einen Tisch mit 10 Stühlen zu setzen?
\item  Jeder Gast soll mit jedem anderen Gast anstoßen. Wie oft klirren die Gläser?
\end{enumerate}
\vspace{4mm}

\solution
\begin{enumerate}[label=(\alph*)]
\item Keiner setzt sich auf zwei Plätze gleichzeitig, daher gibt es keine Wiederholung. Die Reihenfolge macht einen Unterschied. Daher gibt es $\binom{10!}{(10-10)!} = 10!= 3.628.800$ Möglichkeiten. 
\item Niemand stößt mit sich selbst an, daher gibt es keine Wiederholungen. Die Reihenfolge spielt keine Rolle. Die Gläser klirren also $ \begin{pmatrix} 10 \\ 2\end{pmatrix} = 45$ mal.
\end{enumerate}
\vspace{8mm}

\newpage
\exercise
Drei Bits werden über einen digitalen Nachrichtenkanal übertragen. Jedes Bit kann falsch oder richtig empfangen werden.
\begin{enumerate}[label=(\alph*)]
\item Geben Sie den Ereignisraum (Grundmenge) $\Omega$ an.
\item Wie viele Elemente besitzt $\Omega$?
\item Es sei $A_i := \{ \text{ i-tes Bit ist verfälscht}\}$. Geben Sie das Ereignis $A_1$ an.
\end{enumerate}
\vspace{4mm}

\solution
\begin{enumerate}[label=(\alph*)]
\item $\Omega = \{ (b_1, b_2, b_3) \; | \;  b_i \in \{ V (\text{verfälscht}), R (\text{richtig}) \} \land i \in \{1,2,3\} \}$.\\
$\Rightarrow \Omega = \{ (V,V,V), (V,V,R), (V,R,V), (V,R,R), \\(R,V,V), (R,V,R), (R,R,V), (R,R,R) \}$
\item $| \Omega | = Var_3^2(\Omega, m.W) =  2^3 = 8$.
\item $A_1 = \{ (V, b_2, b_3) \; | \;  b_i \in \{ V,R  \}  \land i \in \{2,3\} \} \\
= \{ (V,R,R), (V,V,V), (V,R,V), (V,V,R) ) \}$.
\end{enumerate}
\vspace{8mm}


\exercise
Es sei $\Omega = \{ 1,2,3,4\}$. 
\begin{enumerate}[label=(\alph*)]
\item Welche der folgenden Mengen sind $\sigma$-Algebren?
\begin{align*}
& A =  \{ \emptyset,  \Omega  \} \\
& B =  \{ \emptyset,  \Omega , \{ 1\}, \{ 2,3\}, \{ 4\} \}  \\
& C =  \{ \emptyset,  \Omega ,  \{ 1,2\}, \{ 3, 4\} \}  
\end{align*}
\item Geben Sie die kleinste Sigma-Algebra über $\Omega$ an, in der die Mengen $ \{ 1\}$ und $ \{ 2\}$ enthalten sind.
\end{enumerate}
\vspace{4mm}

\solution
\begin{enumerate}[label=(\alph*)]
\item $A$ ist eine $\sigma$-Algebra, da 
\begin{itemize}
\item $\Omega^c =  \emptyset \in A$
\item $\emptyset^c = \Omega \in A$
\item $\emptyset \cup \Omega = \Omega$
\item $\Omega \in A$
\end{itemize}
\item $B$ ist keine $\sigma$-Algebra, da zum Beispiel $\{ 1 \}^c = \{ 2,3,4 \} \notin B$.
\item $C$ ist eine   $\sigma$-Algebra, da 
\begin{itemize}
\item $\{ 1,2 \}^c = \{ 3,4 \} \in C$
\item $\{ 3,4 \}^c = \{ 1,2 \} \in C$
\item $\Omega \in C$
\item $ \Omega^c =  \emptyset \subset C$
\item $\emptyset^c = \Omega \in C$
\item $\{ 3,4 \} \cup\{ 1,2 \} = \Omega \in C$
\end{itemize}
\item Die kleinste $\sigma$-Algebra ist $  \{ \emptyset,  \Omega , \{ 1\}, \{ 2\}, \{ 1, 2\},  \{ 2,3,4\}, \{1, 3, 4\}, \{ 3,4\} \} $
\end{enumerate}
\vspace{8mm}


\exercise
Bei einer Qualitätskontrolle können Werkstücke zwei Fehler haben, den Fehler $A$ und den Fehler $B$. Aus Erfahrung ist bekannt, dass ein Werkstück mit Wahrscheinlichkeit $0.05$  den Fehler $A$, mit Wahrscheinlichkeit $0.01$ beide Fehler und mit  Wahrscheinlichkeit $0.03$ nur den Fehler $B$ hat.
\begin{enumerate}[label=(\alph*)]
\item Mit welcher Wahrscheinlichkeit hat ein Werkstück den Fehler $B$?
\item Mit welcher Wahrscheinlichkeit ist das Werkstück fehlerfrei, beziehungsweise fehlerhaft?
\item Bei einem Werkstück wurde der Fehler $A$ festgestellt, während die Untersuchung auf Fehler $B$ noch nicht erfolgt ist. Mit welcher Wahrscheinlichkeit hat es auch den Fehler $B$?
\item Mit welcher Wahrscheinlichkeit ist ein Werkstück fehlerfrei, falls es den Fehler $B$ nicht besitzt?
\item Sind die Ereignisse ''Werkstück hat Fehler $A$''  und ''Werkstück hat Fehler $B$'' unabhängig?
\end{enumerate}
\vspace{4mm}

\solution
Gegeben: $P(A) = 0,05; \; P(A \cap B) = 0,01; \; P(A^c \cap B) = 0,03$.
\begin{enumerate}[label=(\alph*)]
\item $P(B) = P(A \cap B) + P(A^c \cap B) = 0,01 + 0,03 = 0,04$.
\item $ \{ \text{Werkstück ist fehlerhaft} \} = A \cup B$.
\begin{align*}
    & P(A \cup B) \\
    & = P((A \cap B) \cup (A \cap B^c) \cup (B \cap A^c))\\
    & = P((A \cap B) \cup (A \cap B^c)) + P(B \cap A^c)\\
    & = P(A) + P(B \cap A^c) \\
    & = 0,05 + 0,03 = 0,08\\
    & \{ \text{Werkstück ist fehlerfrei} \} = A^c \cap B^c.\\
    & P(A^c \cap B^c) = P((A \cup B)^c) = 1 - P(A \cup B)= 0,92
\end{align*}
\item $P(B | A) = \frac{P(B \cap A)}{P(A)} = \frac{0,01}{0,05} = 0,2$.
\item $P(A^c | B^c) = \frac{P(A^c \cap B^c)}{P(B^c)} = \frac{P(A^c \cap B^c)}{1 - P(B)} = \frac{0,92}{0,96} \approx 0,9583$.
\item $0,01 = P(A \cap B) \neq P(A) \cdot P(B) = 0,05 \cdot 0,04 = 0,002$ $$ \Rightarrow A \text{ und } B \text{ sind nicht unabhängig.}$$
\end{enumerate}
\vspace{8mm}

\newpage
\exercise
In einer Fabrik werden die produzierten Werkstücke vor der Auslieferung überprüft. Hierfür werden für jedes Werkstück   zwei Funktionstest durchgeführt.  Die Wahrscheinlichkeit, dass ein Werkstück beide Tests besteht, betrage $0,55$. Die Wahrscheinlichkeit, dass ein Werkstück  den ersten Test besteht betrage $0,72$. 
\begin{enumerate}[label=(\alph*)]
\item Berechnen Sie die Wahrscheinlichkeit, dass ein Werkstück den zweiten Test besteht, wenn er bereits den ersten bestanden hat.
\item Angenommen, die beiden Tests sind stochastisch unabhängig. Wie hoch ist dann die Wahrscheinlichkeit, dass ein Werkstück den zweiten Test besteht?
\end{enumerate}
\vspace{4mm}

\solution
Die Wahrscheinlichkeit für  Test $T_1$  ist $P(T_1) = 0,72$.\\
Die Wahrscheinlichkeit für  Test $T_1$ und Test $T_2$ ist $P(T_1 \cap T_2) = 0,55$.
\begin{enumerate}[label=(\alph*)]
\item $P(T_2 | T_1) = \frac{P(T_1 \cap T_2)}{P(T_1)} = \frac{0,55} {0,72} \approx 0,7639$.
\item Sind $T_1$ und $T_2$ unabhängig, gilt  $P(T_2) = P(T_2 | T_1) = 0,7639$.
\end{enumerate}
\vspace{8mm}


\exercise
In einer Fabrik werden die produzierten Werkstücke vor der Auslieferung überprüft. Hierfür werden für jedes Werkstück  hintereinander zwei Funktionstest durchgeführt.  Die Wahrscheinlichkeit, dass ein Werkstück beide Tests besteht, betrage $0.5$. Die Wahrscheinlichkeit, dass ein Werkstück  den ersten Test besteht betrage $0.6$.  
\begin{enumerate}[label=(\alph*)]
\item Berechnen Sie die Wahrscheinlichkeit, dass ein Werkstück den zweiten Test besteht, wenn er bereits den ersten bestanden hat.
\item Angenommen, die beiden Tests sind stochastisch unabhängig. Wie hoch ist dann die Wahrscheinlichkeit, dass ein Werkstück den zweiten Test besteht?
\end{enumerate}
\vspace{4mm}

\solution
Die Wahrscheinlichkeit für  Test $T_1$  ist $P(T_1) = 0.6$.\\
Die Wahrscheinlichkeit für  Test $T_1$ und Test $T_2$ ist $P(T_1 \cap T_2) = 0.5$.
\begin{enumerate}[label=(\alph*)]
\item $P(T_2 | T_1) = \frac{P(T_1 \cap T_2)}{P(T_1)} = \frac{0.5} {0.6} \approx 0,83$.
\item Sind $T_1$ und $T_2$ unabhängig, gilt  $P(T_2) = P(T_2 | T_1) = 0.83$.
\end{enumerate}
\vspace{8mm}

\newpage
\exercise
In einer Urne befinden sich 4 schwarze und 6 weiße Kugeln.
Es werden nacheinander zwei Kugeln gezogen, wobei die erste Kugel zurückgelegt wird, bevor die Zweite gezogen wird.
Zeigen Sie, dass das Ziehen einer schwarzen oder weißen Kugel im zweiten Zug stochastisch unabhängig davon ist, welche Kugel im ersten Zug gezogen wurde.
Gilt das auch, wenn nach dem ersten Zug die Kugel nicht zurückgelegt wird?
\vspace{4mm}

\solution

Beim Versuch mit Zurücklegen gilt für die bedingten Wahrscheinlichkeiten 
\begin{align*}
    & P(ZW|ES) = \frac{P(ZW \cap ES)}{P(ES)} = \frac{\frac{4}{10} \cdot \frac{6}{10}}{\frac{4}{10}} = \frac{3}{5}\\
    & = P(ZW) = \frac{6}{10} \cdot \frac{6}{10} + \frac{4}{10} \cdot \frac{6}{10} = \frac{3}{5}\\
    & \text{ebenso wie}\\
    & P(ZS|EW) = \frac{P(ZS \cap EW)}{P(EW)} = \frac{\frac{6}{10} \cdot \frac{4}{10}}{\frac{6}{10}} = \frac{2}{5}\\
    & = P(ZS) = \frac{4}{10} \cdot \frac{6}{10} + \frac{4}{10} \cdot \frac{4}{10} = \frac{2}{5} \text{.}
\end{align*}
Damit handelt es sich um stochastisch unabhängige Ereignisse.\\
Beim Versuch ohne Zurücklegen gilt für die bedingten Wahrscheinlichkeiten
\begin{align*}
    & P(ZW|ES) = \frac{P(ZW \cap ES)}{P(ES)} = \frac{\frac{4}{10} \cdot \frac{6}{9}}{\frac{4}{10}} = \frac{2}{3}\\
    & \neq P(ZW) = \frac{6}{10} \cdot \frac{5}{9} + \frac{4}{10} \cdot \frac{6}{9} = \frac{3}{5}\\
    & \text{ebenso wie}\\
    & P(ZS|EW) = \frac{P(ZS \cap EW)}{P(EW)} = \frac{\frac{6}{10} \cdot \frac{4}{9}}{\frac{6}{10}} = \frac{4}{9}\\
    & \neq P(ZS) = \frac{6}{10} \cdot \frac{4}{9} + \frac{4}{10} \cdot \frac{3}{9} = \frac{2}{5} \text{.}
\end{align*}
Damit handelt es sich um keine stochastisch unabhängigen Ereignisse.
\vspace{8mm}

\newpage
\exercise
Geben Sie eine Konstante $c  \in \mathbb{R}$ an, so dass die Funktion 
\begin{align*}
f(x) = \begin{cases} c x^2 \text{ für }  0\leq x \leq 1 \\ 0 \text{ sonst}\end{cases}
\end{align*}
eine Dichte auf $\mathbb{R}$ definiert.
\vspace{4mm}

\solution
Für eine Dichte $f$ auf $\mathbb{R}$ muss gelten: $\int_{\mathbb{R}} f \, d\mu = 1$.\\
\begin{align*}
    & \int_{\mathbb{R}} f \, d \mu = 1\\
    & \Leftrightarrow  \int_{0}^{1} f \, d \mu = 1 \quad \text{, da } f(x) = 0 \text{ für } x \notin [0,1]\\
    & \Leftrightarrow \int_{0}^{1} cx^2 \, dx = 1\\
    & \Leftrightarrow \frac{1}{3}c = 1\\
    & \Leftrightarrow c = 3
\end{align*}
Also ist
\begin{align*}
f(x) = \begin{cases} 3x^2 \text{ für }  0\leq x \leq 1 \\ 0 \text{ sonst}\end{cases}
\end{align*}
eine Dichte auf $\mathbb{R}$.
\vspace{8mm}


\exercise
Geben Sie eine Konstante $c  \in \mathbb{R}$ an, so dass die Funktion 
\begin{align*}
f(x) = \begin{cases} c x^3 \text{ für }  0\leq x \leq 2 \\ 0 \text{ sonst}\end{cases}
\end{align*}
ein Dichte auf $\mathbb{R}$ definiert.
\vspace{4mm}

\solution
Für eine Dichte $f$ auf $\mathbb{R}$ muss gelten: $\int_{\mathbb{R}} f \, d\mu = 1$.\\
\begin{align*}
    & \int_{\mathbb{R}} f \, d \mu = 1\\
    & \Leftrightarrow  \int_{0}^{2} f \, d \mu = 1 \quad \text{, da } f(x) = 0 \text{ für } x \notin [0,2]\\
    & \Leftrightarrow \int_{0}^{2} cx^3 \, dx = 1\\
    & \Leftrightarrow 4c = 1\\
    & \Leftrightarrow c = \frac{1}{4}
\end{align*}
Also ist
\begin{align*}
f(x) = \begin{cases} \frac{1}{4}x^3 \text{ für }  0\leq x \leq 2 \\ 0 \text{ sonst}\end{cases}
\end{align*}
eine Dichte auf $\mathbb{R}$.
\vspace{8mm}


\exercise
Die Zufallsvariablen $X_1$ und $X_2$ seien stochastisch unabhängig   und im Intervall $[0,1]$ gleichverteilt.
Berechnen Sie den Erwartungswert der Zufallsvariablen
\begin{align*}
Y = X_1 \cdot (X_2 - X_1)
\end{align*} 
\vspace{4mm}

\solution
Gleichverteilung auf  $[0,1] \Rightarrow  \mathbb{E}(X_1) =  \mathbb{E}(X_2) = \frac{1}{2}$ und $ \mathbb{E}(X_1^2) = \frac{1}{3}$.\\
Da $X_1$ und $X_2$ unabhängig $\Rightarrow \mathbb{E}(X_1 X_2) =  \mathbb{E}(X_1)  \mathbb{E}(X_2)  = \frac{1}{4}$.\\
Damit ist $$\mathbb{E}(Y) = \mathbb{E}( X_1 \cdot (X_2 - X_1)) \; = \mathbb{E}( X_1 X_2 - X_1^2)  = \mathbb{E}( X_1 X_2) - \mathbb{E}(X_1^2)   =  \frac{1}{4} - \frac{1}{3} = -\frac{1}{12}$$
(für $X \sim U(a,b)$, also $X$ gleichverteilt auf dem Intervall $[a,b]$ ist $\mathbb{E}(X^2) = \frac{1}{3}\frac{b^3-a^3}{b-a}$)
\vspace{8mm}


\exercise
Die Zufallsvariablen $X_1$ und $X_2$ seien stochastisch unabhängig   und im Intervall $[0,2]$ gleichverteilt.
Berechnen Sie den Erwartungswert der Zufallsvariablen
\begin{align*}
Y = 2 \cdot X_1 \cdot X_2 + X_1^2
\end{align*} 
\vspace{4mm}

\solution
Gleichverteilung auf  $[0,2] \Rightarrow  \mathbb{E}(X_1) =  \mathbb{E}(X_2) = 1$  und $ \mathbb{E}(X_1^2) = \frac{4}{3}$.\\
Da $X_1$ und $X_2$ unabhängig $\Rightarrow \mathbb{E}(X_1 X_2) =  \mathbb{E}(X_1)  \mathbb{E}(X_2)  = 1$.  Damit ist $$\mathbb{E}(Y) = \mathbb{E}(  2 \cdot X_1 \cdot X_2 + X_1^2) \; = \mathbb{E}(2 \cdot  X_1 X_2) + \mathbb{E}(X_1^2) =  2 \cdot \mathbb{E}(  X_1 X_2) +  \mathbb{E}(X_1^2) =   2 + \frac{4}{3} = \frac{10}{3}$$
(für $X \sim U(a,b)$, also $X$ gleichverteilt auf dem Intervall $[a,b]$ ist $\mathbb{E}(X^2) = \frac{1}{3}\frac{b^3-a^3}{b-a}$) 
\vspace{8mm}


\exercise
Geben Sie die Axiome für einen Wahrscheinlichkeitsraum an.
\vspace{4mm}

\solution
Ein Wahrscheinlichkeitsraum ist ein Tripel $(\Omega, \mathcal{A}, P) $ bestehend aus der Grundmenge $\Omega  $, einer $\sigma$-Algebra $\mathcal{A} \subset  \mathcal{P}(\Omega)$ und einer Abbildung
$P : \mathcal{A} \to [0,1]$
\begin{align*}
(i) & \; P(\Omega) = 1\\
(ii) & \;  P \biggl(  \bigcup_i A_i  \biggr) = \sum_i P(A_i), \text{ mit } A_i \cap A_j = \emptyset \text{ für } i \neq j
\end{align*}
\vspace{8mm}


\exercise
Formulieren Sie das schwache Gesetz  der großen Zahlen und erläutern Sie die Aussage.
\vspace{4mm}

\solution
Seien $\{X_i  \}_i: \Omega \to \mathbb{R}$ unabhängige, reelle Zufallsvariablen mit $\mathbb{E}(X_i) = \mu < \infty$ und $\mathbb{V}(X_i) = \sigma < \infty$. Dann gilt:
\begin{align*}
P \bigl  ( \bigl | \frac{1}{n} \sum_{i=1}^{n} X_i - \mu \bigr |  \geq \epsilon \bigr) \leq \frac{\sigma}{ n \cdot \epsilon^2} \; \; \underset{n \to \infty}{\longrightarrow} 0
\end{align*}
Das schwache Gesetz der großen Zahlen besagt, dass das arithmetische Mittel  einer großen Stichprobe einer Zufallsvariable mit einer beliebig kleinen, vorgegebenen Wahrscheinlichkeit dem Erwartungswert der Zufallsvariable entspricht.

\textbf{Eigene Alternative:}\\
Werden $n$ unabhängige, aber ansonsten gleichartige Versuche durchgeführt, so liegt der Mittelwert der Ergebnisse bei großem $n$ mit hoher Wahrscheinlichkeit nahe beim Erwartungswert.
\vspace{8mm}

\exercise
Was versteht man unter einem Laplace-Experiment?
\vspace{4mm}

\solution
Man spricht von einem Laplace-Experiment, wenn der Ereignisraum $\Omega$ endlich viele Elemente $\#\Omega$ und ein Ereignis $A \subset \Omega$ die Wahrscheinlichkeit $P(A) = \frac{\#A}{\#\Omega}$ hat.
\vspace{8mm}

\exercise
Erklären Sie die Aussage des zentralen Grenzwertsatzes anhand eines Beispiels.
\vspace{4mm}

\solution
Wenn man den Durchschnitt von $n$ Messungen eines verrauschten Sensor betrachtet, so kann man wegen dem zentralen Grenzwertsatz näherungsweise Annehmen, dass die zugehörige Verteilung Normalverteilt ist.

\textbf{Eigene Alternative:}\\
Die Summe von stochastisch unabhängigen, identisch verteilten Zufallsvariablen ist näherungsweise normalverteilt. Beispiel: Die Wahrscheinlichkeit der Augensumme von $n \to \infty$ Würfeln ist näherungsweise normalverteilt, da jeder Würfel stochastisch unabhängig und gleichverteilt ist.
\vspace{4mm}


\end{document}

