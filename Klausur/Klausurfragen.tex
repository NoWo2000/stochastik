\documentclass[a4paper]{article}
\usepackage[utf8]{inputenc}
\usepackage{amsmath}
%\usepackage{hyperref}
%\usepackage{float}
\usepackage{algorithm}
\usepackage{arevmath}     % For math symbols
\usepackage[noend]{algpseudocode}

\begin{document}
\section{Theorie}
\setlength{\parindent}{0em}

\subsection{Wahrscheinlichkeitsraum}

\subsubsection{Definition}
Ein Wahrscheinlichkeitsraum ist ein Tripel $(\Omega, \mathcal{A}, P)$ bestehend aus der Grundmenge $\Omega$, einer $\sigma$-Algebra $\mathcal{A} \subseteq  \mathcal{P}(\Omega)$ und einer Abbildung
$P : \mathcal{A} \to [0,1]$
\begin{align*}
(i) & \; P(\Omega) = 1 \\
(ii) & \;  P \biggl(  \bigcup_i A_i  \biggr) = \sum_i P(A_i), \text{ mit } A_i \cap A_j = \emptyset \text{ für } i \neq j
\end{align*}
Die Elemente von $\Omega$ werden elementare Ereignisse und die von $\mathcal{A}$ Ereignisse genannt. Mengen M mit $P(M) = 0$ werden Nullmengen genannt.
Die Abbildung $P$ wird Wahrscheinlichkeitsmaß genannt.

\subsubsection{$\sigma$-Algebra}
Es sei $\Omega$ eine Menge und $\mathcal{A} \subset  \mathcal{P}(\Omega)$ ein System von Teilmengen (= Ereignissen). $\mathcal{A}$ heißt $\sigma$-Algebra (Sigma-Algebra) falls gilt:
\begin{align*}
& (i) \; \Omega \in \mathcal{A} \\
& (ii) \; A \in \mathcal{A} \Rightarrow A^c \in \mathcal{A} \\
& (iii) \; A_i \in \mathcal{A} \Rightarrow \bigcup_i A_i \in \mathcal{A} 
\end{align*}
$(A^c = \Omega \setminus A)$

\subsubsection{Diskreter Wahrscheinlichkeitsraum}
Ein diskreter Wahrscheinlichkeitsraum ist ein Wahrscheinlichkeitsraum $(\Omega, \mathcal{A}, P)$, bei dem die Grundmenge $\Omega$ abzählbar ist und die Menge der Ereignisse $\mathcal{A} := \mathcal{P}(\Omega)$ der Potenzmenge entspricht.

\subsubsection{Laplace Experiment}
Ein Laplace-Experiment ist ein Zufallsexperiment bei dem der Ereignisraum $\Omega$ endlich viele Elemente und ein Ereignis $A \subseteq \Omega$ die Wahrscheinlichkeit $P(A) = \frac{\#A}{\#\Omega}$ hat.


\subsection{Bedingte Wahrscheinlichkeit}
Für $A,B \in \mathcal{A}$ und $P(B) > 0$ heißt
\begin{align*}
& P(A \; | \;  B) = \frac{P(A \cap B)}{P(B)} \\
\end{align*}
die bedingte Wahrscheinlichkeit (von $A$ unter $B$).


\subsection{Spamfilter / Satz von Bayes}

\subsubsection{Satz der totalen Wahrscheinlichkeit}
Für eine Zerlegung  $\Omega = \bigcup_{j=1}^{n} B_j, \text{ mit } B_i \cap B_k = \emptyset \text{ für } i \neq k $
\begin{align*}
& P(A ) = \sum_{j=1}^{n}  P(A \; | \;  B_j) \cdot P(B_j)
\end{align*}

\subsubsection{Satz von Bayes}
Für $A,B \in \mathcal{A}$ mit  $P(B) > 0$ gilt
\begin{align*}
& P(A \; | \;  B) = \frac{P(B \; | \; A) \cdot P(A)} {P(B)} \\
\end{align*}

\subsubsection{Stochastische Unabhängigkeit}
Zwei Ereignisse $A,B$ heißen stochastisch unabhängig, falls
\begin{align*}
P(A \cap B) = P(A) \cdot P(B)
\end{align*}
gilt.  Gleichbedeutend damit ist  $P(A | B) = P(A)$ und $P(B  | A) = P(B)$.

\subsubsection{Naiver Bayes'scher Spam Filter}
Gegeben ist eine E-Mail $E$.  Wir möchten anhand des Vorkommens bestimmter Wörter $A_1, \ldots A_n$ in der Mail entscheiden, ob es sich um eine erwünschte Mail $H$ oder eine unerwünschte Mail $S$ (Ham or Spam) handelt. 
(Typische Wörter wären zum Beispiel "reich",  "casino", "Vergrösserung" ...)\\\\
Aus einer Datenbank kann man das Vorkommen dieser Wörter in Spam und Ham Mails zählen und damit empirisch die Wahrscheinlichkeiten $P(A_i | S)$ und $P(A_i | H) $ des Vorkommens dieser Wörter in Spam und Ham Mails ermitteln.  Wir gehen davon aus, dass es sich bei der Mail  prinzipiell mit  Wahrscheinlichkeit $P(E= S) = P(E= H)= \frac{1}{2}$  um eine erwünschte  Mail $H$ oder eine unerwünschte Mail $S$  handeln kann. \\\\
 Wir machen zudem die (naive) Annahme, dass das Vorkommen der Wörter  stochastisch unabhängig ist, also 
\begin{align*}
P(A_1 \cap \cdots \cap A_n | S) = P(A_1 | S) \cdot P(A_2 | S) \cdots P(A_n | S) \\
P(A_1 \cap \cdots \cap A_n | H) = P(A_1 | H) \cdot P(A_2 | H) \cdots P(A_n | H)
\end{align*}
gilt.\\\\
Mit der Formel von Bayes und der totalen Wahrscheinlichkeit  können wir somit berechnen
\begin{align*}
& P(E=S |  A_1 \cap \cdots \cap A_n) = \frac{P(A_1 \cap \cdots \cap A_n | S) \cdot P(S)}{P(A_1 \cap \cdots \cap A_n)} \\
&=  \frac{P(A_1 | S) \cdots P(A_n | S) \cdot P(S)}{P(A_1 \cap \cdots \cap A_n | H) + P(A_1 \cap \cdots \cap A_n | S)} \\
&=  \frac{P(A_1 | S) \cdots P(A_n | S) \cdot P(S)}{P(A_1 | H) \cdots P(A_n | H)  + P(A_1 | S) \cdots P(A_n | S) } \\
\end{align*}
Bemerkung: $P(E=H |  A_1 \cap \cdots \cap A_n) = 1- P(E=S |  A_1 \cap \cdots \cap A_n) $


\subsection{Zufallsvariablen}

\subsubsection{Allgemeine Zufallsvariable}
Sei $(\Omega, \mathcal{A}, P)$ ein Wahrscheinlichkeitsraum und $(\Omega', \mathcal{A}')$ ein Messraum. Eine Zufallsvariable ist eine Abbildung
$$X : \Omega \to \Omega'$$ 
so dass für alle Ereignisse $A' \in  \mathcal{A}'$
$$ X^{-1} (A') \in \mathcal{A}$$
 ein Ereignis in $\mathcal{A}$ ist. Urbilder von Ereignissen sind also Ereignisse.

\subsubsection{Messraum}
Ein Messraum ist ein Paar $(\Omega, \mathcal{A})$ bestehend aus einer Menge $\Omega$ und einer Sigma-Algebra $\mathcal{A} \subset \mathcal{P}(\Omega)$.

\subsubsection{Reelle Zufallsvariable}
Unter einer reellen Zufallsvariable verstehen wir eine Zufallsvariable 
\begin{align*}
& X : \Omega \to \mathbb{R}^n \\
& X(\omega) := \biggl(X_1(\omega), \cdots , X_n(\omega)  \biggr) \; ,
\end{align*}
wobei $(\Omega, \mathcal{A}, P)$ ein Wahrscheinlichkeitsraum ist und $(\mathbb{R}^n, \mathcal{B}(\mathbb{R}^n))$ der $\mathbb{R}^n$ zusammen mit der Borel'schen Sigma-Algebra ist.


\subsection{Erwartungswert}

\subsubsection{Definition}
Für eine reelle, integrierbare Zufallsvariable $X$ ist der Erwartungswert definiert durch
$$ \mathbb{E} (X) := \int_{\Omega} X \; dP \; .$$
Ist $(\Omega, \mathcal{A}, P)$ ein diskreter Wahrscheinlichkeitsraum und $X :\Omega \to \mathbb{R}$ eine eindimensionale reelle Zufallsvariable, so ist
$$ \mathbb{E} (X) = \sum_{\omega \in \Omega}  X(\omega) \cdot P(\omega)$$

\subsubsection{Eigenschaften}
Sind $X,Y : \Omega \to \mathbb{R}^n$   reelle, integrierbare  Zufallsvariablen und $a,b \in \mathbb{R}$ konstant, so gilt:
\begin{align*}
& \mathbb{E}(a \cdot X \pm b \cdot Y) = a \cdot \mathbb{E}(X) \pm b \cdot \mathbb{E}(Y) \\
& \forall x \in \Omega: X(x) \leq Y(x) \;   \Rightarrow \mathbb{E}(X) \leq \mathbb{E}(Y) \\
& X ,Y \text{ stoch. unabhängig} \Rightarrow   \mathbb{E}(X \cdot Y) =  \mathbb{E}(X) \cdot  \mathbb{E}(Y) \\
& \mathbb{E} (1_A) = P (A)
\end{align*}

\subsection{Varianz}
Für eine reelle Zufallsvariable $X$ ist die Varianz definiert durch
$$ \mathbb{V} (X) \; := \;  \mathbb{E}( (X - \mathbb{E}(X))^2) \;  = \; \mathbb{E}(X^2) -  \mathbb{E}(X)^2$$


\subsection{Verteilungen}

\subsubsection{Normalverteilung}
Die Normalverteilung $N{(\mu,\sigma^2)}$ auf $\mathbb{R}$ ist definiert durch
\begin{align*}
& \text{Dichte: } f (x) : = \frac 1{\sigma \sqrt{2\pi}}e^{- \frac {1}{2} (\frac{x- \mu}{ \sigma})^2} \\
&  \Rightarrow \text{Verteilung: } F(x) = N{(\mu,\sigma^2)}(-\infty , x) =  \int_{-\infty}^{x}  \frac 1{\sigma \sqrt{2\pi}}e^{- \frac {1}{2} (\frac{t- \mu}{ \sigma})^2}dt\\
\end{align*}
\textbf{Erwartungswert und Varianz bei $X \sim N(\mu, \sigma^2)$:}
\begin{align*}
& \mathbb{E}(X) = \mu \\
& \mathbb{V}(X) = \sigma^2
\end{align*}

\subsubsection{Verteilungsfunktion}
Für eine reelle Zufallsvariable $X$ heißt 
\begin{align*} 
& F_X : \Omega \to [0,1] \\
& F_X (x) := P (X \leq x) := P_X (( -\infty, x )) = P(X^{-1} (-\infty, x))
\end{align*}
Verteilungsfunktion von $X$.

\subsubsection{Gleichverteilung}
Die Gleichverteilung $U{(a,b)}$ auf einem Intervall $(a,b) \subset \mathbb{R}$ ist definiert durch
\begin{align*}
& \text{Dichte: } f (x) : = \frac{1_{(a,b)}}{|b-a| } \\
& \text{Verteilung: } F (x) =  P_f( (-\infty, x))  =  \int_{-\infty}^{x} \frac{1_{(a,b)}}{|b-a|} dt\\\
& = \begin {cases} 0 \text{ für } x \leq a \\   \frac{x-a}{|b-a|} \text{ für } a \leq x \leq b \\ 1 \text{ für }  x \geq b \\  \end{cases}
\end{align*}
\textbf{Erwartungswert und Varianz bei $X \sim U(a,b)$:}
\begin{align*}
& \mathbb{E}(X) = \frac{a+b}2 \\
& \mathbb{V}(X) = \mathbb{E}(X^2) - \left({\mathbb{E}(X)} \right)^2  = \frac{1}{3}\frac{b^3  - a^3}{b - a} - \left( {\frac{a + b}{2}} \right)^2 \\
    &= \frac{1}{12}(b - a)^2
\end{align*}

\subsubsection{Dichte}
Sei $\Omega \subset \mathbb{R}^n$ und $(\Omega, \mathcal{A})$ ein Messraum, wobei alle $A \in \mathcal{A}$ Lebesgue-messbar sind.
 Eine Funktion $f: \Omega \to \mathbb{R}$ heißt Dichte, falls für ihr Lebesgue-Integral $\int_{\Omega} f d \mu = 1$ gilt.


\subsection{Schwaches Gesetz der großen Zahlen}

\subsubsection{Definition}
Seien $X_i : \Omega \to \mathbb{R}$ unabhängige, reelle Zufallsvariablen mit $\mathbb{E}(X_i) = \mu < \infty$ und $\mathbb{V}(X_i) = \sigma < \infty$, dann gilt
\begin{align*}
P \bigl  ( \bigl | \frac{1}{n} \sum_{i=1}^{n} X_i - \mu \bigr |  \geq \epsilon \bigr) \leq \frac{\sigma}{ n \cdot \epsilon^2} \; \; \underset{n \to \infty}{\longrightarrow} 0
\end{align*}
(stochastische Konvergenz). 

\subsubsection{Bedeutung}
Das schwache Gesetz der großen Zahlen besagt, dass das arithmetische Mittel einer großen Stichprobe einer Zufallsvariable mit einer beliebig kleinen Wahrscheinlichkeit dem Erwartungswert der Zufallsvariable entspricht.\\

\textbf{Gegenteilige (äquivalente) Formulierung:}\\
Die Wahrscheinlichkeit, dass die Differenz zwischen beobachteter relativer Häufigkeit und theoretischer Wahrscheinlichkeit kleiner ist als eine beliebig kleine positive Zahl $\epsilon$, ist für eine unendlich große Stichprobe praktisch $1$.


\subsection{Zentraler Grenzwertsatz}

\subsubsection{Definition}
Sei $(\Omega, \mathcal{A}, P)$ ein Wahrscheinlichkeitsraum und $X_n :  \Omega \to \mathbb{R}$  eine folge stochastisch unabhängiger, identisch verteilter, reeller Zufallsvariablen mit $\mathbb{E}(X_n) = \mu$ und $\mathbb{V}(X_n)= \sigma^2$. Dann gilt für das arithmetische Mittel $S_n:= \frac{1}{n} \sum_{i=1}^n X_i$
\begin{align*}
P_{ \frac{\sqrt{n}}{\sigma} (S_n-\mu)} \to P_{N(0,1)}
\end{align*}
wobei $ P_{N(0,1)}$ das Wahrscheinlichkeits-Maß mit der Dichte $ \frac {1}{ \sqrt{2\pi}}e^{- \frac {1}{2} x^2}$ ist.
\subsubsection{Bedeutung}
Die Summe von $n$ identisch verteilten, stochastisch unabhängigen Zufallsvariablen ist näherungsweise normalverteilt.\\

\textbf{Beispiel Würfel:}\\
Die Augensumme von $n \to \infty$ Würfeln ist normalverteilt, wenn alle Würfel von einander stochastisch unabhängig und gleichverteilt sind.





\end{document}
