\documentclass[a4paper,13pt]{scrartcl}

\usepackage[utf8]{inputenc}
\usepackage{amsfonts}
\usepackage{amsmath}
\usepackage{amssymb}
\usepackage{amsthm}
\usepackage{enumitem}
\usepackage{hyperref}
\usepackage[all]{hypcap}
\hypersetup{
        colorlinks = true, % comment this to make xdvi work
        linkcolor  = blue,
        citecolor  = red,
        filecolor  = Gold,
        urlcolor   = blue,
        pdfborder  = {0 0 0} 
}
\usepackage{color}
\usepackage[ngerman]{babel}
\usepackage[pdftex]{graphicx}
%\usepackage[T1]{fontenc}
\usepackage{graphicx}
\pagestyle{empty}

%\topmargin20mm
\oddsidemargin0mm
\parindent0mm
\parskip2mm
\textheight25.5cm
\textwidth15.8cm
\unitlength1mm

\newcommand{\exercise}{\vspace*{0.2cm}
\stepcounter{aufgabe}
\noindent
\textbf{Aufgabe \arabic{aufgabe}}: }
\newcounter{aufgabe}



\begin{document}
\section*{\large  Aufgaben Übungsblätter}
\hrule
\hrule
\vspace{4mm}

\exercise
Beim Lottospiel werden ohne Zurücklegen $6$ Zahlen aus $49$ gezogen. Berechnen Sie die folgenden Wahrscheinlichkeiten:
\begin{enumerate}[label=(\alph*)]
\item Alle $6$ Gewinnzahlen richtig zu tippen.
\item Genau $5$ richtige Gewinnzahlen zu tippen.
\item Mindestens $3$ richtige Gewinnzahlen zu tippen.
\item Alle $6$ Gewinnzahlen sind gerade.
\end{enumerate}
\vspace{8mm}

\exercise
In einem Raum gibt es acht Lampen, die unabhängig voneinander ein- und ausgeschaltet werden können.
Wie viele Beleuchtungsarten gibt es
\begin{enumerate}[label=(\alph*)]
\item wenn fünf Lampen brennen sollen?
\item wenn mindestens fünf Lampen brennen sollen?
\end{enumerate}
\vspace{8mm}

\exercise
Sie geben eine Party und laden 10 Leute ein.
\begin{enumerate}[label=(\alph*)]
\item Wie viele Möglichkeiten gibt es, die 10 Gäste an einen Tisch mit 10 Stühlen zu setzen?
\item  Jeder Gast soll mit jedem anderen Gast anstoßen. Wie oft klirren die Gläser?
\end{enumerate}
\vspace{8mm}

\exercise
Drei Bits werden über einen digitalen Nachrichtenkanal übertragen. Jedes Bit kann falsch oder richtig empfangen werden.
\begin{enumerate}[label=(\alph*)]
\item Geben Sie den Ereignisraum (Grundmenge) $\Omega$ an.
\item Wie viele Elemente besitzt $\Omega$?
\item Es sei $A_i := \{ \text{ i-tes Bit ist verfälscht}\}$. Geben Sie das Ereignis $A_1$ an.
\end{enumerate}
\vspace{8mm}

\exercise
Es sei $\Omega = \{ 1,2,3,4\}$. 
\begin{enumerate}[label=(\alph*)]
\item Welche der folgenden Mengen sind $\sigma$-Algebren?
\begin{align*}
& A =  \{ \emptyset,  \Omega  \} \\
& B=  \{ \emptyset,  \Omega , \{ 1\}, \{ 2,3\}, \{ 4\} \}  \\
& C=  \{ \emptyset,  \Omega ,  \{ 1,2\}, \{ 3, 4\} \}  
\end{align*}
\item Geben Sie die kleinste Sigma-Algebra über $\Omega$ an, in der die Mengen $ \{ 1\}$ und $ \{ 2\}$ enthalten sind.
\end{enumerate}
\vspace{8mm}

\exercise
Bei einer Qualitätskontrolle können Werkstücke zwei Fehler haben, den Fehler $A$ und den Fehler $B$. Aus Erfahrung ist bekannt, dass ein Werkstück mit Wahrscheinlichkeit $0.05$  den Fehler $A$, mit Wahrscheinlichkeit $0.01$ beide Fehler und mit  Wahrscheinlichkeit $0.03$ nur den Fehler $B$ hat.
\begin{enumerate}[label=(\alph*)]
\item Mit welcher Wahrscheinlichkeit hat ein Werkstück den Fehler $B$?
\item Mit welcher Wahrscheinlichkeit ist das Werkstück fehlerfrei, beziehungsweise fehlerhaft?
\item Bei einem Werkstück wurde der Fehler $A$ festgestellt, während die Untersuchung auf Fehler $B$ noch nicht erfolgt ist. Mit welcher Wahrscheinlichkeit hat es auch den Fehler $B$?
\item Mit welcher Wahrscheinlichkeit ist ein Werkstück fehlerfrei, falls es den Fehler $B$ nicht besitzt?
\item Sind die Ereignisse ''Werkstück hat Fehler $A$''  und ''Werkstück hat Fehler $B$'' unabhängig?
\end{enumerate}
\vspace{8mm}

\exercise
In einer Fabrik werden die produzierten Werkstücke vor der Auslieferung überprüft. Hierfür werden für jedes Werkstück   zwei Funktionstest durchgeführt.  Die Wahrscheinlichkeit, dass ein Werkstück beide Tests besteht, betrage $0,55$. Die Wahrscheinlichkeit, dass ein Werkstück  den ersten Test besteht betrage $0,72$. 
\begin{enumerate}[label=(\alph*)]
\item Berechnen Sie die Wahrscheinlichkeit, dass ein Werkstück den zweiten Test besteht, wenn er bereits den ersten bestanden hat.
\item Angenommen, die beiden Tests sind stochastisch unabhängig. Wie hoch ist dann die Wahrscheinlichkeit, dass ein Werkstück den zweiten Test besteht?
\end{enumerate}
\vspace{8mm}

\exercise
In einer Fabrik werden die produzierten Werkstücke vor der Auslieferung überprüft. Hierfür werden für jedes Werkstück  hintereinander zwei Funktionstest durchgeführt.  Die Wahrscheinlichkeit, dass ein Werkstück beide Tests besteht, betrage $0.5$. Die Wahrscheinlichkeit, dass ein Werkstück  den ersten Test besteht betrage $0.6$.  
\begin{enumerate}[label=(\alph*)]
\item Berechnen Sie die Wahrscheinlichkeit, dass ein Werkstück den zweiten Test besteht, wenn er bereits den ersten bestanden hat.
\item Angenommen, die beiden Tests sind stochastisch unabhängig. Wie hoch ist dann die Wahrscheinlichkeit, dass ein Werkstück den zweiten Test besteht?
\end{enumerate}
\vspace{8mm}

\exercise
In einer Urne befinden sich 4 schwarze und 6 weiße Kugeln.
Es werden nacheinander zwei Kugeln gezogen, wobei die erste Kugel zurückgelegt wird, bevor die Zweite gezogen wird.
Zeigen Sie, dass das Ziehen einer schwarzen oder weißen Kugel im zweiten Zug stochastisch unabhängig davon ist, welche Kugel im ersten Zug gezogen wurde.
Gilt das auch, wenn nach dem ersten Zug die Kugel nicht zurückgelegt wird?
\vspace{8mm}

\exercise
Geben Sie eine Konstante $c  \in \mathbb{R}$ an, so dass die Funktion 
\begin{align*}
f(x) = \begin{cases} c x^2 \text{ für }  0\leq x \leq 1 \\ 0 \text{ sonst}\end{cases}
\end{align*}
eine Dichte auf $\mathbb{R}$ definiert.
\vspace{8mm}

\exercise
Geben Sie eine Konstante $c  \in \mathbb{R}$ an, so dass die Funktion 
\begin{align*}
f(x) = \begin{cases} c x^3 \text{ für }  0\leq x \leq 2 \\ 0 \text{ sonst}\end{cases}
\end{align*}
ein Dichte auf $\mathbb{R}$ definiert.
\vspace{8mm}

\exercise
Die Zufallsvariablen $X_1$ und $X_2$ seien stochastisch unabhängig   und im Intervall $[0,1]$ gleichverteilt.
Berechnen Sie den Erwartungswert der Zufallsvariablen
\begin{align*}
Y = X_1 \cdot (X_2 - X_1)
\end{align*} 
\vspace{8mm}

\exercise
Die Zufallsvariablen $X_1$ und $X_2$ seien stochastisch unabhängig   und im Intervall $[0,2]$ gleichverteilt.
Berechnen Sie den Erwartungswert der Zufallsvariablen
\begin{align*}
Y = 2 \cdot X_1 \cdot X_2 + X_1^2
\end{align*} 
\vspace{8mm}

\exercise
Geben Sie die Axiome für einen Wahrscheinlichkeitsraum an.
\vspace{8mm}

\exercise
Formulieren Sie das schwache Gesetz  der grossen Zahlen und erläutern Sie die Aussage.
\vspace{8mm}

\exercise
Was versteht man unter einem Laplace-Experiment?
\vspace{8mm}

\exercise
Erklären Sie die Aussage des zentralen Grenzwertsatzes anhand eines Beispiels.
\vspace{8mm}

\end{document}

